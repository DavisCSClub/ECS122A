\section{Case Study: Rod Cutting}

\subsection*{Problem Statement}
Given a rod of length $n$ and a set of prices $P = \{ p_1, p_2, ... p_n \}$ such that $p_i$ denotes the price of a piece of rod with length $i$, find the optimal (maximum)  revenue $r_i$ for cutting the rod into pieces whose length sum to $n$.

\subsection*{Steps}
\begin{enumerate}
	\item Start from rod length = 1.
	\item With each subrod length, there will always be one ``cut" -- splitting the rod into a left half and right half. If the cut is equivalent to the length of the subrod, then it means that the entire length of the subrod was used (Left half will have the full length and the right half will have zero length). 
	\item Iterate from rod length = 1 to rod length = n.\\
		On each iteration of length $i$:
	\begin{enumerate}
		\item Assume that the left half has the full length and the right half has length 0.
		\item Decrement the left half's left by 1 and increase the right half's length by 1.
		\item Sum the revenue of the left half with the price of the right half.
		\item Repeat this process until the left half is of length 0.
		\item The maximum of all these sums become the maximum revenue of length $i$.
	\end{enumerate}
\end{enumerate}

\newpage

\subsection{Example}
\textbf{Given a rod of length = 8 and P defined as $\{ 1, 5, 8, 9, 10, 17, 17, 20 \}$, find the maximum revenue.}

\begin{table}[H]
	\centering
	\begin{tabular}{| c | c | c | c | c | c | c | c | c |}
		\hline
		Length
			&	1
			&	2
			&	3
			&	4
			&	5
			&	6
			&	7
			&	8\\
		\hline
		Price
			&	1
			&	5
			&	8
			&	9
			&	10
			&	17
			&	17
			&	20\\
		\hline
	\end{tabular}
\end{table}

\subsubsection*{Subrod Length = 1}

\begin{table}[h]
	\centering
	\begin{tabular}{| c | c | c |}
		\hline
		Length(Left)	&	Length(Right)	&	Revenue(Left) + Price(Right)\\
		\hline
		0
			&	1
			&	0 + 1 = 1\\
		\hline
	\end{tabular}	
\end{table}

\begin{table}[H]
	\centering
	\begin{tabular}{| c | c | c | c | c | c | c | c | c |}
		\hline
		Length
		&	1
		&	2
		&	3
		&	4
		&	5
		&	6
		&	7
		&	8\\
		\hline
		Price
		&	1
		&	5
		&	8
		&	9
		&	10
		&	17
		&	17
		&	20\\
		\hline
		Revenue
		&	1
		&	
		&	
		&
		&	
		&	
		&	
		&	\\
		\hline
	\end{tabular}
\end{table}

\subsubsection*{Subrod Length = 2}

\begin{table}[h]
	\centering
	\begin{tabular}{| c | c | c |}
		\hline
		Length(Left)	&	Length(Right)	&	Revenue(Left) + Price(Right)\\
		\hline
		1
			&	1
			&	1 + 1 = 2\\
		\hline
		0	
			&	2
			&	0 + 5 = 5\\
		\hline
	\end{tabular}	
\end{table}

\begin{table}[H]
	\centering
	\begin{tabular}{| c | c | c | c | c | c | c | c | c |}
		\hline
		Length
		&	1
		&	2
		&	3
		&	4
		&	5
		&	6
		&	7
		&	8\\
		\hline
		Price
		&	1
		&	5
		&	8
		&	9
		&	10
		&	17
		&	17
		&	20\\
		\hline
		Revenue
		&	1
		&	5
		&	
		&
		&	
		&	
		&	
		&	\\
		\hline
	\end{tabular}
\end{table}

\subsubsection*{Subrod Length = 3}

\begin{table}[h]
	\centering
	\begin{tabular}{| c | c | c |}
		\hline
		Length(Left)	&	Length(Right)	&	Revenue(Left) + Price(Right)\\
		\hline
		2
		&	1
		&	5 + 1 = 6\\
		\hline
		1	
		&	2
		&	1 + 5 = 6\\
		\hline
		0
		&	3
		&	0 + 8 = 8\\
		\hline
	\end{tabular}	
\end{table}

\begin{table}[H]
	\centering
	\begin{tabular}{| c | c | c | c | c | c | c | c | c |}
		\hline
		Length
		&	1
		&	2
		&	3
		&	4
		&	5
		&	6
		&	7
		&	8\\
		\hline
		Price
		&	1
		&	5
		&	8
		&	9
		&	10
		&	17
		&	17
		&	20\\
		\hline
		Revenue
		&	1
		&	5
		&	8
		&
		&	
		&	
		&	
		&	\\
		\hline
	\end{tabular}
\end{table}

\newpage

\subsubsection*{Subrod Length = 4}

\begin{table}[H]
	\centering
	\begin{tabular}{| c | c | c |}
		\hline
		Length(Left)	&	Length(Right)	&	Revenue(Left) + Price(Right)\\
		\hline
		3
		&	1
		&	8 + 1 = 9\\
		\hline
		2	
		&	2
		&	5 + 5 = 10\\
		\hline
		1
		&	3
		&	1 + 8 = 9\\
		\hline
		0
		&	4
		&	0 + 9 = 9\\
		\hline
	\end{tabular}	
\end{table}

\begin{table}[H]
	\centering
	\begin{tabular}{| c | c | c | c | c | c | c | c | c |}
		\hline
		Length
		&	1
		&	2
		&	3
		&	4
		&	5
		&	6
		&	7
		&	8\\
		\hline
		Price
		&	1
		&	5
		&	8
		&	9
		&	10
		&	17
		&	17
		&	20\\
		\hline
		Revenue
		&	1
		&	5
		&	8
		&	10
		&	
		&	
		&	
		&	\\
		\hline
	\end{tabular}
\end{table}

\subsubsection*{Subrod Length = 5}

\begin{table}[H]
	\centering
	\begin{tabular}{| c | c | c |}
		\hline
		Length(Left)	&	Length(Right)	&	Revenue(Left) + Price(Right)\\
		\hline
		4
		&	1
		&	10 + 1 = 11\\
		\hline
		3	
		&	2
		&	8 + 5 = 13\\
		\hline
		2
		&	3
		&	5 + 8 = 13\\
		\hline
		1
		&	4
		&	1 + 9 = 10\\
		\hline
		0
		&	5
		&	0 + 10 = 10\\
		\hline
	\end{tabular}	
\end{table}

\begin{table}[H]
	\centering
	\begin{tabular}{| c | c | c | c | c | c | c | c | c |}
		\hline
		Length
		&	1
		&	2
		&	3
		&	4
		&	5
		&	6
		&	7
		&	8\\
		\hline
		Price
		&	1
		&	5
		&	8
		&	9
		&	10
		&	17
		&	17
		&	20\\
		\hline
		Revenue
		&	1
		&	5
		&	8
		&	10
		&	13
		&	
		&	
		&	\\
		\hline
	\end{tabular}
\end{table}

\subsubsection*{Subrod Length = 6}

\begin{table}[H]
	\centering
	\begin{tabular}{| c | c | c |}
		\hline
		Length(Left)	&	Length(Right)	&	Revenue(Left) + Price(Right)\\
		\hline
		5
		&	1
		&	13 + 1 = 14\\
		\hline
		4	
		&	2
		&	10 + 5 = 15\\
		\hline
		3
		&	3
		&	8 + 8 = 16\\
		\hline
		2
		&	4
		&	5 + 9 = 14\\
		\hline
		1
		&	5
		&	1 + 10 = 11\\
		\hline
		0
		&	6
		&	0 + 17 = 17\\
		\hline
	\end{tabular}	
\end{table}

\begin{table}[H]
	\centering
	\begin{tabular}{| c | c | c | c | c | c | c | c | c |}
		\hline
		Length
		&	1
		&	2
		&	3
		&	4
		&	5
		&	6
		&	7
		&	8\\
		\hline
		Price
		&	1
		&	5
		&	8
		&	9
		&	10
		&	17
		&	17
		&	20\\
		\hline
		Revenue
		&	1
		&	5
		&	8
		&	10
		&	13
		&	17
		&	
		&	\\
		\hline
	\end{tabular}
\end{table}

\subsubsection*{Subrod Length = 7}

\begin{table}[H]
	\centering
	\begin{tabular}{| c | c | c |}
		\hline
		Length(Left)	&	Length(Right)	&	Revenue(Left) + Price(Right)\\
		\hline
		6
		&	1
		&	17 + 1 = 18\\
		\hline
		5
		&	2
		&	13 + 5 = 18\\
		\hline
		4
		&	3
		&	10 + 8 = 18\\
		\hline
		3
		&	4
		&	8 + 9 = 17\\
		\hline
		2
		&	5
		&	5 + 10 = 15\\
		\hline
		1
		&	6
		&	1 + 17 = 18\\
		\hline
		0
		&	7
		&	0 + 17 = 17\\
		\hline
	\end{tabular}	
\end{table}

\begin{table}[H]
	\centering
	\begin{tabular}{| c | c | c | c | c | c | c | c | c |}
		\hline
		Length
		&	1
		&	2
		&	3
		&	4
		&	5
		&	6
		&	7
		&	8\\
		\hline
		Price
		&	1
		&	5
		&	8
		&	9
		&	10
		&	17
		&	17
		&	20\\
		\hline
		Revenue
		&	1
		&	5
		&	8
		&	10
		&	13
		&	17
		&	18
		&	\\
		\hline
	\end{tabular}
\end{table}

\subsubsection*{Subrod Length = 8}

\begin{table}[H]
	\centering
	\begin{tabular}{| c | c | c |}
		\hline
		Length(Left)	&	Length(Right)	&	Revenue(Left) + Price(Right)\\
		\hline
		7
		&	1
		&	18 + 1 = 19\\
		\hline
		6
		&	2
		&	17 + 5 = 22\\
		\hline
		5
		&	3
		&	13 + 8 = 21\\
		\hline
		4
		&	4
		&	10 + 9 = 19\\
		\hline
		3
		&	5
		&	8 + 10 = 18\\
		\hline
		2
		&	6
		&	5 + 17 = 22\\
		\hline
		1
		&	7
		&	1 + 17 = 18\\
		\hline
		0
		&	8
		&	0 + 20 = 20\\
		\hline
	\end{tabular}	
\end{table}

\begin{table}[H]
	\centering
	\begin{tabular}{| c | c | c | c | c | c | c | c | c |}
		\hline
		Length
		&	1
		&	2
		&	3
		&	4
		&	5
		&	6
		&	7
		&	8\\
		\hline
		Price
		&	1
		&	5
		&	8
		&	9
		&	10
		&	17
		&	17
		&	20\\
		\hline
		Revenue
		&	1
		&	5
		&	8
		&	10
		&	13
		&	17
		&	18
		&	22\\
		\hline
	\end{tabular}
\end{table}

$$
\text{The maximum revenue for a rod of length 8 is 22.}
$$