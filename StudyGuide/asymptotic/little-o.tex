\section{o-Notation (Little O)}

\subsection*{Notation}
$$
f(n) \in o(g(n))
$$

\subsection*{Formal Definition}
For a given function $g(n)$, $o(g(n))$ is the set  of functions for which every positive constant $c > 0$, there exists a constant $n_0 > 0$ such that $0 \leq f(n) \leq c \cdot g(n)$ for all $n \geq n_0$.
$$
o(g(n)) = \{ f(n) : \exists \text{ } n_0 \text{ s.t. } 0 \leq f(n) \leq c \cdot g(n) \text{ } \forall \text{ } n \geq n_0, c \geq 0 \}
$$

\subsection*{Informal Definition}
The function g(n) is an upper bound that is not asymptotically tight. For all positive constant values of $c$, there must exists a constant $n_0$ such that $0 \leq f(n) \leq c \cdot g(n)$ for all $n \geq n_0$. The value of $n_0$ may not depend on n, but may depend on $c$.\\\\
Another way to perceive Little O notation is that for $f(n) \in o(g(n))$, the function $f$'s asymptotic growth is strictly less than that of the function $g$'s. In this sense, Little O can be seen as a ``stronger" bound in comparison to Big O. By proving that a function is an element of Little O, it also proves that the function is an element of Big O.

\subsection*{Limit Definition}
$$
\lim\limits_{n\to\infty} \frac{f(n)}{g(n)} = 0
$$

\subsection{Example}
\textbf{Prove that $f(n) = 2n$ has an upper bound $o(n^2)$.}
\begin{eqnarray*}
	0 \leq c \cdot g(n) &\leq& f(n) \text{ for } n \geq n_0\\
	0 \leq c \cdot 2n &\leq& n^2 \text{ for } n \geq n_0\\
	2c &\leq& n \text{ for } n \geq n_0\\
	2c &\leq& n_0
\end{eqnarray*}
For Little O to hold true, the inequality needs to hold true for all $c > 0$ and for all $n > n_0$. From simplifying the inequality, we assert that the inequality will hold true as long as the value of $n_0$ is twice the value of $c$. Given that they are both constants, then there exists a constant value of $n_0$ for all positive constant $c$ that fulfill this inequality.\\\\
Another method to solve this problem is to use the limit definition.
\begin{eqnarray*}
	&\lim\limits_{n\to\infty}& \frac{2n}{n^2}\\
	&\lim\limits_{n\to\infty}& \frac{2}{n} = 0	
\end{eqnarray*}

\newpage

\subsection{Example}
\textbf{Prove that $f(n) = 2n^2$ does not have the upper bound $o(n^2)$.}
\begin{eqnarray*}
	0 \leq c \cdot g(n) &\leq& f(n) \text{ for } n \geq n_0\\
	0 \leq c \cdot 2n^2 &\leq& n^2 \text{ for } n \geq n_0\\
	2c &\leq& 1\text{ for } n \geq n_0
\end{eqnarray*}
For a function to have the Little O bound, the inequality must hold true for all positive $c$. However, simplification of the inequality asserts that the inequality will only hold true for all $c < \frac{1}{2}$. Therefore, $f(n) = 2n^2$ does not have the upper bound $o(n^2)$.