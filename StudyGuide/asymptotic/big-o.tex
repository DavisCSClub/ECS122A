\section{O-Notation (Big O)}

\subsection*{Notation}
$$f(n) \in O(g(n))$$

\subsection*{Formal Definition}
For a given function $g(n)$, $O(g(n))$ is the set of functions for which there exists positive constants $c$ and $n_0$ such that $0 \leq f(n) \leq c \cdot g(n)$ for all $n \geq n_0$.
$$
O(g(n)) = \{ f(n) : \exists \text{ } c, n_0 \text{ s.t. } 0 \leq f(n) \leq c \cdot g(n) \text{ } \forall \text{ } n \geq n_0 \}
$$

\subsection*{Informal Definition}
The function $g(n)$ is an asymptotic upper bound for the function $f(n)$ if there exists constants $c$ and $n_0$ such that $0 \leq f(n) \leq c \cdot g(n)$ for $n \geq n_0$.\\\\
Another way to perceive Big O notation is that for $f(n) \in O(g(n))$, the function $f$'s asymptotic\footnote{Asymptotic: As given variable approaches infinity.} growth is no faster than that of function $g$'s.

\subsection*{Limit Definition}
$$ 
\lim\limits_{n \to \infty} \frac{f(n)}{g(n)} < \infty
$$

\subsection{Example}
\textbf{Prove that asymptotic upper bound of $f(n) = 2n+10$ is $g(n) = n^2$}.
\begin{eqnarray*}
	0 \leq f(n) &\leq& c \cdot g(n) \text{ for } n \geq n_0\\
	0 \leq 2n + 10 &\leq& c \cdot n^2 \text{ for } n \geq n_0
\end{eqnarray*}
Arbitrarily choose $c$ and $n_0$ values. Simplest is to turn one of the variables into the value $1$ and solve. For this example, we will assign the value 1 to $n_0$.
\begin{eqnarray*}
	0 \leq 2n + 10 &\leq& c \cdot n^2 \text{ for } n \geq 1\\
	2(1) + 10 &\leq& c \cdot (1)^2\\
	12 &\leq& c
\end{eqnarray*}
By picking $n_0 = 1$ and $c = 12$, the inequality of $2n+10 \leq 12n^2$ will hold true for all $n \geq 1$. Since there exists a constant $c$ and $n_0$ that fulfill this inequality, we have proven that $f(n) = 2n+10 = O(n^2)$.