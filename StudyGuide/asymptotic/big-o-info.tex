\subsection{O-Notation (Big O)}
\textbf{Notation:} 
$$f(n) \in O(g(n))$$
\textbf{Formal Definition:}\\
For a given function $g(n)$, $O(g(n))$ is the set of functions for which there exists positive constants $c$ and $n_0$ such that $0 \leq f(n) \leq c \cdot g(n)$ for all $n \geq n_0$.
$$
O(g(n)) = \{ f(n) : \exists \text{ } c, n_0 \text{ s.t. } 0 \leq f(n) \leq c \cdot g(n) \text{ } \forall \text{ } n \geq n_0 \}
$$
\textbf{Informal Definition:}\\
The function $g(n)$ is an asymptotic upper bound for the function $f(n)$ if there exists constants $c$ and $n_0$ such that $0 \leq f(n) \leq c \cdot g(n)$ for $n \geq n_0$.\\\\
Another way to perceive Big O notation is that for $f(n) \in O(g(n))$, the function $f$'s asymptotic\footnote{Asymptotic: As given variable approaches infinity.} growth is no faster than that of function $g$'s.\\\\
\textbf{Limit Definition:}
$$ 
\lim\limits_{n \to \infty} \frac{f(n)}{g(n)} < \infty
$$