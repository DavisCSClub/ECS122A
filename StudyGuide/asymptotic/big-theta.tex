\section{$\Theta$-notation (Big Theta)}

\subsection*{Notation}
$$
f(n) \in \Theta(g(n))
$$

\subsection*{Formal Definition}
For a given function $g(n)$, $\Theta(g(n))$ is the set of functions for which there exists positive constants $c_1$, $c_2$, and $n_0$ such that $0 \leq c_1 \cdot g(n) \leq f(n) \leq c_2 \cdot g(n)$ for all $n \geq n_0$.
$$
\Theta(g(n)) = \{ f(n) : \exists \text{ } c_1, c_2, n_0 \text{ s.t. } 0 \leq c_1 \cdot g(n) \leq f(n) \leq c_2 \cdot g(n) \text{ } \forall \text{ } n \geq n_0 \}
$$

\subsection*{Informal Definition}
The function g(n) is an asymptotic tight bound for the function f(n) if there exists constants $c_1$, $c_2$, and $n_0$ such that $0 \leq c_1 \cdot g(n) \leq f(n) \leq c_2 \cdot g(n)$ for $n \geq n_0$.\\\\
Big theta implies that $f(n) = O(g(n))$ and $f(n) = \Omega(g(n))$.

\subsection*{Limit Definition}
$$
\lim\limits_{n\to\infty} \frac{f(n)}{g(n)} \in \mathbb{R}_{>0}
$$