\section{Master Theorem}
Used for divide and conquer recurrences that follow the generic form:
$$
T(n) = a \cdot T(\frac{n}{b}) + f(n) \text{ where } a \geq 1, b > 1
$$

\subsection{Case 1}

\subsubsection*{Condition}
$$
f(n) \in O(n^c)
$$
$$
c < log_b(a)
$$

\subsubsection*{Solution}
$$
T(n) = \Theta(n^{log_b(a)})
$$

\subsection{Case 2}

\subsubsection*{Condition}
$$
f(n) \in \Theta(n^c)
$$
$$
c = log_b(a)
$$

\subsubsection*{Solution}
$$
T(n) \in \Theta(n^{log_b(a)} \cdot log_2(n) )
$$

\subsection{Case 3}

\subsubsection*{Condition}
$$
f(n) \in \Omega(n^c)
$$
$$
c > log_b(a)
$$

\subsubsection*{Regularity Condition}
This case must also fulfill the regularity condition.
$$
a \cdot f(\frac{n}{b}) \leq k \cdot f(n) \text{ where } k < 1
$$

\subsubsection*{Solution}
$$
T(n) = \Theta(f(n))
$$

\subsubsection*{Remark}
The idea behind this case is that given the generic form, the function $f(n)$ will grow far quicker than $a \cdot T(\frac{n}{b})$ and will be the primary influence of $T(n)$'s asymptotic behavior.

\pagebreak

\subsection{Example}
$$
T(n) = 64T(\frac{n}{4}) + 1000n^2
$$

\subsubsection*{Given}
$$f(n) = 1000n^2 \in \Theta(n^2)$$
$$a = 64$$
$$b = 4$$
$$c = 2$$

\subsubsection*{Condition}
\begin{eqnarray*}
	c &\text{ ? }& log_b(a)\\
	2 &\text{ ? }& log_4(64)\\
	2 &<& 3
\end{eqnarray*}
$$\text{Condition satisfied for case 1}$$

\subsection*{Solution}
$$
\therefore T(n) = \Theta(n^{log_4(64)}) = \Theta(n^3)
$$

\subsection{Example}
$$
T(n) = 32T(\frac{n}{2}) + 20n^5
$$

\subsubsection*{Given}
$$f(n) = 20n^5 \in \Theta(n^5)$$
$$a = 32$$
$$b = 2$$
$$c = 5$$

\subsubsection*{Condition}
\begin{eqnarray*}
	c &\text{ ? }& log_b(a)\\
	5 &\text{ ? }& log_2(32)\\
	5 &=& 5
\end{eqnarray*}
$$\text{Condition satisfied for case 2}$$

\subsection*{Solution}
$$
\therefore T(n) = \Theta(n^{log_2(32)} \cdot log_2(n)) = \Theta(n^5 \cdot lg(n))
$$

\subsection{Example}
$$
T(n) = 7T(\frac{n}{7}) + 19n^{11}
$$

\subsubsection*{Given}
$$f(n) = 19n^{11} \in \Theta(n^{11})$$
$$a = 7$$
$$b = 7$$
$$c = 11$$

\subsubsection*{Condition}
\begin{eqnarray*}
	c &\text{ ? }& log_b(a)\\
	11 &\text{ ? }& log_7(7)\\
	5 &>& 1
\end{eqnarray*}
$$\text{Condition partially fulfilled for case 3. Must also check regularity condition.}$$
\begin{eqnarray*}
	a \cdot f(\frac{n}{b}) &\leq& k \cdot f(n)\\
	7 \cdot \left[ 19(\frac{n}{7})^{11} \right] &\leq& k \cdot 19n^{11}\\
	7 \cdot \frac{n^{11}}{7^{11}} &\leq& k \cdot n^{11}\\
	\frac{1}{7^{10}} \cdot n^{11} &\leq& k \cdot n^{11}
\end{eqnarray*}
$$\text{Choosing } k = \frac{1}{7^{10}} < 1 \text{ fulfills the regularity condition.}$$

\subsection*{Solution}
$$
\therefore T(n) = \Theta(19n^{11})
$$
